%%%%%%%%%%%%%%%%%%%%%%%%%%%%%%%%%%%%%%%%%%%%%%%%%%
% Basic setup. Most papers should leave these options alone.
\documentclass[ceqn,usenatbib,onecolumn]{mnras}
% fleqn aligns equations to the left.
% to centre align replace fleqn with ceqn.


\usepackage[utf8]{inputenc}
\usepackage{enumerate}
\usepackage{natbib}

\title{JPbib2019 Bibliography}
\author{John Proctor}
\date{April 2019}

\begin{document}
\maketitle

\section{Purpose}
This document is used to maintain the master version of the bibliography file \texttt{JPbib2019.bib}. Sections are generated for new entries added each month. Usefull notes describing the relevance of the citation are included.
\par Other Overleaf projects should access this master version of \texttt{JPbib2019.bib}. Remember to refresh the linked version in your project.

%%%%%%%%%%%%%%%%%%%%%%%%%%%%%%%%%%%%%%%%%%%%%%%%%%%
\section{May 2019}
\subsection{week 21}
\begin{itemize}
    \item Peter Toft's Radon transform thesis \citet{7910dc8d5b654c90ac4bc94c67d06f01}
    \item Obtained the correct reference to the MaNGA survey overview \citet{2015ApJ...798....7B}.
\end{itemize}


\subsection{Week 18}
\begin{itemize}
    \item \citet{2017MNRAS.472.1401A} : {Massive post-starburst galaxies at z $>$ 1 are compact proto-spheroids} : We investigate the relationship between the quenching of star formation and the structural transformation of massive galaxies, using a large sample of photometrically selected post-starburst galaxies in the UKIDSS Ultra-Deep Survey field. We find that post-starburst galaxies at high redshift (z $>$ 1) show high Sérsic indices, significantly higher than those of active star-forming galaxies, but with a distribution that is indistinguishable from the old quiescent population. We conclude that the morphological transformation occurs before (or during) the quenching of star formation.  Our findings are consistent with a scenario in which massive passive galaxies are formed from three distinct phases: (1) gas-rich dissipative collapse to very high densities, forming the proto-spheroid, (2) rapid quenching of star formation to create the 'red nugget' with post-starburst features and (3) a gradual growth in size as the population ages, perhaps as a result of minor mergers.
    \item 
\end{itemize}

%%%%%%%%%%%%%%%%%%%%%%%%%%%%%%%%%%%%%%%%%%%%%%%%%%%
\section{April 2019}
\subsection{Week 17}
\par \citet{2019ApJ...872...76N} : {Accurate Identification of Galaxy Mergers with Imaging} : Since mass ratio has the largest effect on the classification, we create separate classification approaches for minor and major mergers that can be applied to SDSS imaging or adapted for other imaging surveys.
\par \citet{2019arXiv190100856W} : {The Data Analysis Pipeline for the SDSS-IV MaNGA IFU Galaxy Survey: Overview} : This paper describes the first public release of the DAP software and assesses its output provided by the recent SDSS Data Release 15 (DR15). The DAP has focused on measurements that are as close as possible to the data and require minimal model-based interpretations or assumptions for reproduction. For DR15, these measurements include stellar kinematics (velocity and velocity dispersion), emission-line properties (kinematics, fluxes, and equivalent widths), and spectral indices (e.g., D4000 and the Lick indices). We provide an overview of how these measurements are made in the MaNGA DAP, including discussion of its software design, workflow, performance, and output data model. \textbf{Section 7 goes into particular depth with regard to the assessments of the
stellar kinematics}. The python code used to produce this plot and many others
in this paper can be found at \url{https://github.com/sdss/mangadap/tree/master/docs/papers/Overview/scripts}.
\par \citet{2019MNRAS.483..172D} : {SDSS-IV MaNGA: signatures of halo assembly in kinematically misaligned galaxies} : We investigate the relationship of kinematically misaligned galaxies with their large-scale environment, in the context of halo assembly bias. According to numerical simulations, halo age at fixed halo mass is intrinsically linked to the large-scale tidal environment created by the cosmic web. We investigate the relationship between distances to various cosmic web features and present-time gas accretion rate. We select a sub-sample of $\sim$900 central galaxies from the MaNGA survey with \textbf{defined global position angles (PA; angle at which velocity change is greatest) for their stellar and H$\alpha$ gas components} up to a minimum of 1.5 effective radii (Re). : The emphasis is on gas-disc PA misalignment. To identify accreting galaxies, we estimate the two-dimensional (2D) global PA of the stellar and H$\alpha$ gas velocity fields using the FIT\_KINEMATIC\_PA routine outlined in Krajnović et al. (2006). By default this finds the angle corresponding to the bisecting line that has greatest velocity change along it (i.e. the angle of peak rotational velocity). The offset angle between kinematic components is defined as  
$$\Delta{PA}=|PA_{stellar}−PAH\alpha|$$
We define galaxies with $\Delta$PA $\ge 30$ degrees to be significantly kinematically misaligned. An example of an aligned and a misaligned galaxy is shown in Fig. 2.
\par \citet{1996ApJ...466..104Z} : {The Environment of ``E+A'' Galaxies} : The spectrum of an "E + A" galaxy (Dressier \& Gunn) which is dominated by a young stellar component but lacks the emission lines characteristic of any significant, on-going star formation suggests that the galaxy experienced a brief, powerful starburst within the last gigayear (Dressler \& Gunn; Couch \& Sharples). We conclude that interactions with the cluster environment, in the form of the intracluster medium or cluster potential, are not essential for "E+A" formation and therefore that the presence of these galaxies in distant clusters does not provide strong evidence for the effects of cluster environment on galaxy evolution. If one mechanism is responsible for "E+A" formation, then the observations that "E+A"s exist in the field and that at least five of the 21 in our sample have clear tidal features argue that galaxy-galaxy interactions and mergers are that mechanism. 
\par \citet{2017MNRAS.466..798C} : {Improving the full spectrum fitting method: accurate convolution with Gauss-Hermite functions} : Penalized Pixel-Fitting (pPXF) - \textbf{Extract the stellar or gas kinematics and stellar population from galaxy spectra via full spectrum fitting}.

\par \citet{2004PASP..116..138C} : {Parametric Recovery of Line-of-Sight Velocity Distributions from Absorption-Line Spectra of Galaxies via Penalized Likelihood} : The dynamics of stars in a galaxy is uniquely defined by the distribution function and the potential in which the stars move. Considering galaxies as pure stellar systems, the spectrum observed at a certain sky position is a (luminosity‐weighted) sum of individual stellar spectra redshifted according to their LOS velocities. If one makes the assumption that the spectrum of all stars is given by a single template, then it simply reduces to the convolution between that spectrum and the LOSVD, which can then be retrieved by solving the inverse problem, i.e., deconvolving the spectra using the template. 
\par \citet{2005MNRAS.357..937G} : {266 E+A galaxies selected from the Sloan Digital Sky Survey Data Release 2: the origin of E+A galaxies}. E+A galaxies are characterized as galaxies with strong Balmer absorption lines but without any [OII] or H$\alpha$ emission lines. The existence of strong Balmer absorption lines indicates that E+A galaxies have experienced starburst within the past one gigayear. However, the lack of [OII] and H$\alpha$ emission lines indicates that E+A galaxies do not have any on-going star formation. Therefore, E+A galaxies are interpreted as post-starburst galaxies. For many years, however, it has been a mystery why E+A galaxies started starburst and why they quenched star formation abruptly. 
\par \citet{10.1093/pasj/55.4.771} : {H$\delta$-Strong Galaxies in the Sloan Digital Sky Survey: I. The Catalog} : We present here a new and homogeneous sample of 3340 galaxies selected from the Sloan Digital Sky Survey (SDSS) \textbf{based solely on the observed strength of their H$\delta$ hydrogen Balmer absorption line}. The presence of a strong H$\delta$ line within the spectrum of a galaxy indicates that the galaxy has undergone a significant change in its star-formation history within the last Gigayear. : \textbf{H$\delta$-strong (HDS) galaxies}.
\subsection{Week 16}
\par \citet{10.1093/mnras/stv2878} : {\textbf{Shape asymmetry:} a morphological indicator for automatic detection of galaxies in the post-coalescence merger stages} : We present a \textbf{new morphological indicator} designed for automated recognition of galaxies with faint asymmetric tidal features suggestive of an ongoing or past merger. We use the new indicator, together with pre-existing diagnostics of galaxy structure to study the role of galaxy mergers in inducing (post-) starburst spectral signatures in local galaxies, and investigate whether (post-) starburst galaxies play a role in the build-up of the ‘red sequence’.
\par \citet{2015A&A...582A..21B} : {Tracing kinematic (mis)alignments in CALIFA merging galaxies. Stellar and ionized gas kinematic orientations at every merger stage}. The aim of this work is to analyze the stellar and ionized gas velocity distributions as the merger event evolves by studying several galaxies at different stages of this event; we also want to compare kinematic properties of these galaxies with those derived from a set of non-interacting objects. 
\par \citet{2014MNRAS.438.1038R}: {Caught in the act: cluster `k+a' galaxies as a link between spirals and S0s}: Gentler mechanisms, such as ram-pressure stripping or weak galaxy-galaxy interactions, appear to be responsible for ending star formation in these intermediate-redshift cluster disc galaxies.

\par \citet{2011MNRAS.414.2923K} : {The ATLAS$^{3D}$ project - II. Morphologies, kinemetric features and alignment between photometric and kinematic axes of early-type galaxies}
\par \citet{2007MNRAS.382..960K} : {The UV properties of E+A galaxies: constraints on feedback-driven quenching of star formation}
\par \citet{2018MNRAS.480.2217S} : {SDSS-IV MaNGA: characterizing non-axisymmetric motions in galaxy velocity fields using the Radon transform} :  We demonstrate the \textbf{Radon transform} by applying it to gas and stellar velocity fields from the first $\sim$2800 galaxies of the SDSS-IV MaNGA IFU survey. We separately classify gas and stellar velocity fields into five categories based on the shape of their radial PA$_k$ profiles. 
\par \citet{2003MNRAS.344.1000B} : {Stellar population synthesis at the resolution of 2003} : We present a new model for computing the spectral evolution of stellar populations... The model reproduces in detail typical galaxy spectra from the Early Data Release (EDR) of the Sloan Digital Sky Survey (SDSS). This model should be particularly useful for interpreting the spectra gathered by modern spectroscopic surveys in terms of constraints on the star formation histories and metallicities of galaxies.
\par \citet{2000ApJ...529..886C} : {The Asymmetry of Galaxies: Physical Morphology for Nearby and High-Redshift Galaxies} : Describes how measured asymmetry can be an indicator of galaxy interaction or mergers.

\subsection{Week 15}
\par \citet{2012MNRAS.420..672S} : {From star-forming spirals to passive spheroids: integral field spectroscopy of E+A galaxies} : Abstract: We present three-dimensional spectroscopy of 11 E+A galaxies at z= 0.06-0.12. These galaxies were selected for their strong H$\delta$ [JP: 4102\AA] absorption but weak (or non-existent) [O II] $\lambda$3727 and H$\alpha$ emission. This selection suggests that a recent burst of star formation was triggered but subsequently abruptly ended. The analysis uses H-delta equivalent widths. A stars cover 33 percent of the galaxy image. 

\par \citet{1997A&A...325.1025P} : {Indicators of star formation: 4000 {\r{A}} break and Balmer lines.} : Dn4000 and equivalent width of Balmer lines. Describes the application of D$_n$4000 and H$\delta$ spectral indices as an indication of recent star formation. 
\par \citet{2004ApJ...609..683T} : {Field E+A Galaxies at Intermediate Redshifts ($0.3 < z < 1$)} : PSB galaxies.
\par \citet{2009MNRAS.395..144W} : {Post-starburst galaxies: more than just an interesting curiosity} : {Post-starburst galaxies in the VVDS : PCA of the $4000\AA$ break.} : Are gas-rich mergers an important mechanism for the build-up of the red sequence since $z\sim1$.
\par \citet{2004ApJ...608..752B} : {Nearly 5000 Distant Early-Type Galaxies in COMBO-17: A Red Sequence and Its Evolution since $z\sim1$} : Galaxy colour bi-modality
\par \citet{1972ApJ...176....1G} : {On the Infall of Matter Into Clusters of Galaxies and Some Effects on Their Evolution} : Ram pressure stripping.
\par \citet{2006MNRAS.366..787K} : {Kinemetry: a generalization of photometry to the higher moments of the line-of-sight velocity distribution} : The wealth of features seen in stellar kinematic maps of early-type galaxies (Emsellem et al. 2004) confirms the usefulness of two-dimensional data, but also poses a problem to efficiently harvest and interpret the important features from the maps.
\subsection{Week 14}
\par \citet{2011ApJ...742...11S} The Interstellar Medium in Distant Star-forming Galaxies: Turbulent Pressure, Fragmentation, and Cloud Scaling Relations in a Dense Gas Disk at z = 2.3.
\par \citet{brian_cherinka_2018_1146705} : sdss/marvin: Marvin Beta 2.2.0
\par \citet{2008ApJ...682..231S} : {Kinemetry of SINS High-Redshift Star-Forming Galaxies: Distinguishing Rotating Disks from Major Mergers}.
\par \citet{2016A&A...591A..85B} : {Distinguishing disks from mergers: Tracing the kinematic asymmetries in local (U)LIRGs using kinemetry-based criteria}.

\section{March 2019}
\citet{2016MNRAS.463..832W} : The evolution of post-starburst galaxies from z=2 to 0.5
\par \citet{2017MNRAS.472.1401A} : Massive post-starburst galaxies at $z \ge 1$ are compact proto-spheroids

\section{February 2019}
\citet{Cappellari2008} : {Measuring the inclination and mass-to-light ratio of
        axisymmetric galaxies via anisotropic Jeans models of stellar
        kinematics}
\par \citet{Bundy_2014} : {{OVERVIEW} {OF} {THE} {SDSS}-{IV} {MaNGA} {SURVEY}: {MAPPING} {NEARBY} {GALAXIES} {AT} {APACHE} {POINT} {OBSERVATORY}}


%%%%%%%%%%%%%%%%%%%% REFERENCES %%%%%%%%%%%%%%%%%%
% The best way to enter references is to use BibTeX:
\bibliographystyle{mnras}
\bibliography{JPbib2019} 

\end{document}
