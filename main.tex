%%%%%%%%%%%%%%%%%%%%%%%%%%%%%%%%%%%%%%%%%%%%%%%%%%
% Basic setup. Most papers should leave these options alone.
\documentclass[ceqn,usenatbib,onecolumn]{mnras}
% fleqn aligns equations to the left.
% to centre align replace fleqn with ceqn.


\usepackage[utf8]{inputenc}
\usepackage{enumerate}
\usepackage{natbib}

\title{JPbib Bibliography 2019}
\author{John Proctor}
\date{April 2019}

\begin{document}
\maketitle

\section{Purpose}
This document is used to maintain the master version of the bibliography file \texttt{JPbib2019.bib}. Sections are generated for new entries added each month.
Other Overleaf projects will access this master version of \texttt{JPbib2019.bib}. Remember to refresh the linked version in your project.

\section{April 2019}
\subsection{Week 15}
\par \citet{1997A&A...325.1025P} : {Indicators of star formation: 4000 {\r{A}} break and Balmer lines.} : Dn4000 and equivalent width of Balmer lines. Describes the application of D$_n$4000 and H$\delta$ spectral indices as an indication of recent star formation. 
\par \citet{2004ApJ...609..683T} : {Field E+A Galaxies at Intermediate Redshifts ($0.3 < z < 1$)} : PSB galaxies.
\par \citet{2009MNRAS.395..144W} : {Post-starburst galaxies: more than just an interesting curiosity} : {Post-starburst galaxies in the VVDS : PCA of the $4000\AA$ break.} : Are gas-rich mergers an important mechanism for the build-up of the red sequence since $z\sim1$.
\par \citet{2004ApJ...608..752B} : {Nearly 5000 Distant Early-Type Galaxies in COMBO-17: A Red Sequence and Its Evolution since $z\sim1$} : Galaxy colour bi-modality
\par \citet{1972ApJ...176....1G} : {On the Infall of Matter Into Clusters of Galaxies and Some Effects on Their Evolution} : Ram pressure stripping.
\par \citet{2006MNRAS.366..787K} : {Kinemetry: a generalization of photometry to the higher moments of the line-of-sight velocity distribution}
\subsection{Week 14}
\par \citet{2011ApJ...742...11S} The Interstellar Medium in Distant Star-forming Galaxies: Turbulent Pressure, Fragmentation, and Cloud Scaling Relations in a Dense Gas Disk at z = 2.3.
\par \citet{brian_cherinka_2018_1146705} : sdss/marvin: Marvin Beta 2.2.0
\par \citet{2008ApJ...682..231S} : {Kinemetry of SINS High-Redshift Star-Forming Galaxies: Distinguishing Rotating Disks from Major Mergers}.
\par \citet{2016A&A...591A..85B} : {Distinguishing disks from mergers: Tracing the kinematic asymmetries in local (U)LIRGs using kinemetry-based criteria}.

\section{March 2019}
\citet{2016MNRAS.463..832W} : The evolution of post-starburst galaxies from z=2 to 0.5
\par \citet{2017MNRAS.472.1401A} : Massive post-starburst galaxies at $z \ge 1$ are compact proto-spheroids

\section{February 2019}
\citet{Cappellari2008} : {Measuring the inclination and mass-to-light ratio of
        axisymmetric galaxies via anisotropic Jeans models of stellar
        kinematics}
\par \citet{Bundy_2014} : {{OVERVIEW} {OF} {THE} {SDSS}-{IV} {MaNGA} {SURVEY}: {MAPPING} {NEARBY} {GALAXIES} {AT} {APACHE} {POINT} {OBSERVATORY}}


%%%%%%%%%%%%%%%%%%%% REFERENCES %%%%%%%%%%%%%%%%%%
% The best way to enter references is to use BibTeX:
\bibliographystyle{mnras}
\bibliography{JPbib2019} 

\end{document}
