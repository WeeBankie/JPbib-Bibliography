%%%%%%%%%%%%%%%%%%%%%%%%%%%%%%%%%%%%%%%%%%%%%%%%%%
% Basic setup. Most papers should leave these options alone.
\documentclass[ceqn,usenatbib,onecolumn]{mnras}
% fleqn aligns equations to the left.
% to centre align replace fleqn with ceqn.


\usepackage[utf8]{inputenc}
\usepackage{enumerate}
\usepackage{natbib}

\title{JPbib2019 Bibliography}
\author{John Proctor}
\date{April 2019}

\begin{document}
\maketitle

\section{Purpose}
This document is used to maintain the master version of the bibliography file \texttt{JPbib2019.bib}. Sections are generated for new entries added each month. Useful notes describing the relevance of the citation are included.
\par Other Overleaf projects should access this master version of \texttt{JPbib2019.bib}. Remember to refresh the linked version in your project.

%%%%%%%%%%%%%%%%%%%%%%%%%%%%%%%%%%%%%%%%%%%%%%%%%%%

\section{July 2019}
\subsection{Week 27}

\begin{itemize}
    \item \citet{2013MNRAS.429.2212M} : {Towards a physical picture of star formation quenching: the photometric properties of recently quenched galaxies in the Sloan Digital Sky Survey} : We select a sample of young passive galaxies from the Sloan Digital Sky Survey data release 7 in order to study the processes that quench star formation in the local universe. Quenched galaxies are identified based on the contribution of A-type stars to their observed (central) spectra and relative lack of ongoing star formation; we find that such systems account for roughly 2.5 per cent of all galaxies with log ( $M/M_{odot }) ge 9.5$, and have a space density of ̃2.2 × 10-4 Mpc-3. We show that quenched galaxies span a range of morphologies, but visual classifications suggest that they are predominantly early-type systems. Their visual early-type classification is supported by quantitative structural measurements (Sérsic indices) that show a notable lack of disc-dominated galaxies, suggesting that any morphological transformation associated with galaxies' transition from star forming to passive - e.g. the formation of a stellar bulge - occurs contemporaneously with the decline of their star formation activity. We show that there is no clear excess of optical active galactic nuclei (AGN) in quenched galaxies, suggesting that: (i) AGN feedback is not associated with the majority of quenched systems or (ii) that the observability of quenched galaxies is such that the quenching phase in general outlives any associated nuclear activity. Comparison with classical post-starburst galaxies shows that both populations show similar signatures of bulge growth, and we suggest that the defining characteristic of post-starburst galaxies is the efficiency of their bulge growth rather than a particular formation mechanism.
\end{itemize}

\section{June 2019}
\subsection{Week 26}

\begin{itemize}
    \item \citet{2001AJ....122.1861S} Color Separation of Galaxy Types in the Sloan Digital Sky Survey Imaging Data : use in the CMD section.
    \item \citet{2003ApJS..149..289B} The Optical and Near-Infrared Properties of Galaxies. I. Luminosity and Stellar Mass Functions: use in the CMD section.
    \item \citet{2003ApJ...585L...5H} The Overdensities of Galaxy Environments as a Function of Luminosity and Color : use in the CMD section.
\end{itemize}

\subsection{Week 25}

\begin{itemize}
    \item \citet{2019ApJS..240...23A} The Fifteenth Data Release of the Sloan Digital Sky Surveys: First Release of MaNGA-derived Quantities, Data Visualization Tools, and Stellar Library : DR15 : Twenty years have passed since first light for the Sloan Digital Sky Survey (SDSS). Here, we release data taken by the fourth phase of SDSS (SDSS-IV) across its first three years of operation (2014 July–2017 July). This is the third data release for SDSS-IV, and the 15th from SDSS (Data Release Fifteen; DR15). New data come from MaNGA—we release 4824 data cubes, as well as the first stellar spectra in the MaNGA Stellar Library (MaStar), the first set of survey-supported analysis products (e.g., stellar and gas kinematics, emission-line and other maps) from the MaNGA Data Analysis Pipeline, and a new data visualization and access tool we call "Marvin." The next data release, DR16, will include new data from both APOGEE-2 and eBOSS; those surveys release no new data here, but we document updates and corrections to their data processing pipelines. The release is cumulative; it also includes the most recent reductions and calibrations of all data taken by SDSS since first light. In this paper, we describe the location and format of the data and tools and cite technical references describing how it was obtained and processed. The SDSS website (www.sdss.org) has also been updated, providing links to data downloads, tutorials, and examples of data use. Although SDSS-IV will continue to collect astronomical data until 2020, and will be followed by SDSS-V (2020–2025), we end this paper by describing plans to ensure the sustainability of the SDSS data archive for many years beyond the collection of data.
    \item \citet{Mutch_2011} : THE MID-LIFE CRISIS OF THE MILKY WAY AND M31 : Upcoming next generation galactic surveys, such as Gaia and HERMES, will deliver unprecedented detail about the structure and make-up of our Galaxy, the Milky Way, and promise to radically improve our understanding of it. However, to benefit our broader knowledge of galaxy formation and evolution we first need to quantify how typical the Galaxy is with respect to other galaxies of its type. Through modeling and comparison with a large sample of galaxies drawn from the Sloan Digital Sky Survey and Galaxy Zoo, we provide tentative yet tantalizing evidence to show that both the Milky Way and nearby M31 are undergoing a critical transformation of their global properties. Both appear to possess attributes that are consistent with galaxies midway between the distinct blue and red bimodal color populations. In extragalactic surveys, such "green valley" galaxies are transition objects whose star formation typically will have all but extinguished in less than 5 Gyr. This finding reveals the possible future of our own galactic home and opens a new window of opportunity to study such galactic transformations up close.
\end{itemize}

\subsection{week 24}

\begin{itemize}
    \item \citet{2016ApJS..224...38A} : Shocked POststarbust Galaxy Survey. I. Candidate Post-starbust Galaxies with Emission Line Ratios Consistent with Shocks : There are many mechanisms by which galaxies can transform from blue, star-forming spirals, to red, quiescent early-type galaxies, but our current census of them does not form a complete picture. Recent observations of nearby case studies have identified a population of galaxies that quench “quietly.” Traditional poststarburst searches seem to catch galaxies only after they have quenched and transformed, and thus miss any objects with additional ionization mechanisms exciting the remaining gas. The Shocked POststarburst Galaxy Survey (SPOGS) aims to identify transforming galaxies, in which the nebular lines are excited via shocks instead of through star formation processes. Utilizing the Oh-Sarzi-Schawinski-Yi (OSSY) measurements on the Sloan Digital Sky Survey Data Release 7 catalog, we applied Balmer absorption and shock boundary criteria to identify 1067 SPOG candidates (SPOGs*) within z = 0.2. SPOGs* represent 0.2\% of the OSSY sample galaxies that exceed the continuum signal-to-noise cut (and 0.7\% of the emission line galaxy sample). SPOGs* colors suggest that they are in an earlier phase of transition than OSSY galaxies that meet an “E+A” selection. SPOGs* have a 13\% 1.4 GHz detection rate from the Faint Images of the Radio Sky at Twenty Centimeters Survey, higher than most other subsamples, and comparable only to low-ionization nuclear emission line region hosts, suggestive of the presence of active galactic nuclei (AGNs). SPOGs* also have stronger Na I D absorption than predicted from the stellar population, suggestive of cool gas being driven out in galactic winds. It appears that SPOGs* represent an earlier phase in galaxy transformation than traditionally selected poststarburst galaxies, and that a large proportion of SPOGs* also have properties consistent with disruption of their interstellar media, a key component to galaxy transformation. It is likely that many of the known pathways to transformation undergo a SPOG phase. Studying this sample of SPOGs* further, including their morphologies, AGN properties, and environments, has the potential for us to build a more complete picture of the initial conditions that can lead to a galaxy evolving.
    \item \citet{2019NatAs...3..440P} : The diverse evolutionary pathways of post-starburst galaxies : About 35 years ago a class of galaxies with unusually strong Balmer absorption lines and weak emission lines was discovered in distant galaxy clusters1,2. These objects, alternatively referred to as post-starburst, E+A or k+a galaxies, are now known to occur in all environments and at all redshifts, with many exhibiting compact morphologies and low-surface brightness features indicative of past galaxy mergers. They are commonly thought to represent galaxies that are transitioning from blue to red sequence, making them critical to our understanding of the origins of galaxy bimodality. However, recent observational studies have questioned this simple interpretation. From observations alone, it is challenging to disentangle the different mechanisms that lead to the quenching of star formation in galaxies. Here we present examples of three different evolutionary pathways that lead to galaxies with strong Balmer absorption lines in the Evolution and Assembly of GaLaxies and their Environments (EAGLE) simulation: classical blue-red quenching, blue-blue cycle and red-red rejuvenation. The first two are found in both post-starburst galaxies and galaxies with truncated star formation. Each pathway is consistent with scenarios hypothesized for observational samples. The fact that ‘post-starburst’ signatures can be attained via various evolutionary channels explains the diversity of observed properties, and lends support to the idea that slower quenching channels are important at low redshift.
    \item \citet{2018MNRAS.477.1708P} The origins of post-starburst galaxies at z \&lt; 0.05 : Post-starburst galaxies can be identified via the presence of prominent Hydrogen Balmer absorption lines in their spectra. We present a comprehensive study of the origin of strong Balmer lines in a volume-limited sample of 189 galaxies with 0.01 < z < 0.05, log(M⋆/M⊙)>9.5 and projected axial ratio b/a > 0.32. We explore their structural properties, environments, emission lines, and star formation histories, and compare them to control samples of star-forming and quiescent galaxies, and simulated galaxy mergers. Excluding contaminants, in which the strong Balmer lines are most likely caused by dust-star geometry, we find evidence for three different pathways through the post-starburst phase, with most events occurring in intermediate-density environments: (1) a significant disruptive event, such as a gas-rich major merger, causing a starburst and growth of a spheroidal component, followed by quenching of the star formation (70 per cent of post-starburst galaxies at 9.5<log({M}⋆/{M}⊙)<10.5 and 60 per cent at log({M}⋆/{M}⊙)>10.5⁠); (2) at 9.5<log({M}⋆/{M}⊙)<10.5⁠, stochastic star formation in blue-sequence galaxies, causing a weak burst and subsequent return to the blue sequence (30 per cent); (3) at log({M}⋆/{M}⊙)>10.5⁠, cyclic evolution of quiescent galaxies which gradually move towards the high-mass end of the red sequence through weak starbursts, possibly as a result of a merger with a smaller gas-rich companion (40 per cent). Our analysis suggests that active galactic nuclei (AGNs) are ‘on’ for 50 per cent of the duration of the post-starburst phase, meaning that traditional samples of post-starburst galaxies with strict emission-line cuts will be at least 50 per cent incomplete due to the exclusion of narrow-line AGNs.
    \item \citet{2006MNRAS.373..469B} Galaxy bimodality versus stellar mass and environment: We analyse a z < 0.1 galaxy sample from the Sloan Digital Sky Survey focusing on the variation in the galaxy colour bimodality with stellar mass and projected neighbour density $\Sigma$, and on measurements of the galaxy stellar mass functions. The characteristic mass increases with environmental density from about 1010.6 to (Kroupa initial mass function, H0 = 70) for $\Sigma$ in the range 0.1-10Mpc-2. The galaxy population naturally divides into a red and blue sequence with the locus of the sequences in colour-mass and colour-concentration indices not varying strongly with environment. The fraction of galaxies on the red sequence is determined in bins of 0.2 in log$\Sigma$ and bins). The red fraction fr generally increases continuously in both $\Sigma$ and such that there is a unified relation: . Two simple functions are proposed which provide good fits to the data. These data are compared with analogous quantities in semi-analytical models based on the Millennium N-body simulation: the Bower et al. and Croton et al. models that incorporate active galactic nucleus feedback. Both models predict a strong dependence of the red fraction on stellar mass and environment that is qualitatively similar to the observations. However, a quantitative comparison shows that the Bower et al. model is a significantly better match; this appears to be due to the different treatment of feedback in central galaxies.
    \item \citet{0004-637X-600-2-681}: Quantifying the Bimodal Color-Magnitude Distribution of Galaxies: We analyze the bivariate distribution, in color versus absolute magnitude ( u-r vs. M r ), of a low-redshift sample of galaxies from the Sloan Digital Sky Survey (2400 deg 2 , 0.004 < z < 0.08, -23.5 < M r < -15.5). We trace the bimodality of the distribution from luminous to faint galaxies by fitting double Gaussians to the color functions separated in absolute magnitude bins. Color-magnitude (CM) relations are obtained for red and blue distributions (early- and late-type, predominantly field, galaxies) without using any cut in morphology. Instead, the analysis is based on the assumption of normal Gaussian distributions in color. We find that the CM relations are well fitted by a straight line plus a tanh function. Both relations can be described by a shallow CM trend (slopes of about -0.04, -0.05) plus a steeper transition in the average galaxy properties over about 2 mag. The midpoints of the transitions ( M r = -19.8 and -20.8 for the red and blue distributions, respectively) occur around 2 × 10 10 ##IMG## [http://ej.iop.org/icons/Entities/calM.gif] {Script M} ☉ after converting luminosities to stellar mass. Separate luminosity functions are obtained for the two distributions. The red distribution has a more luminous characteristic magnitude and a shallower faint-end slope ( M * = -21.5, α = -0.8) compared to the blue distribution (α ≈ -1.3, depending on the parameterization). These are approximately converted to galaxy stellar mass functions. The red distribution galaxies have a higher number density per magnitude for masses greater than about 3 × 10 10 ##IMG## [http://ej.iop.org/icons/Entities/calM.gif] {Script M} ☉ . Using a simple merger model, we show that the differences between the two functions are consistent with the red distribution being formed from major galaxy mergers.
    \item \citet{2017A&A...597A..48F} Stellar kinematics across the Hubble sequence in the CALIFA survey: general properties and aperture corrections: We present the stellar kinematic maps of a large sample of galaxies from the integral-field spectroscopic survey CALIFA. The sample comprises 300 galaxies displaying a wide range of morphologies across the Hubble sequence, from ellipticals to late-type spirals. This dataset allows us to homogeneously extract stellar kinematics up to several effective radii. \textbf{In this paper, we describe the level of completeness of this subset of galaxies withrespect to the full CALIFA sample, as well as the virtues and limitations of the kinematic extraction compared to other well-known integral-field surveys}. In addition, we provide averaged integrated velocity dispersion radial profiles for different galaxy types, which are particularly useful to apply aperture corrections for single aperture measurements or poorly resolved stellar kinematics of high-redshift sources. The work presented in this paper sets the basis for the study of more general properties of galaxies that will be explored in subsequent papers of the survey.
    \item \citet{2015A&A...582A..21B} Tracing kinematic (mis)alignments in CALIFA merging galaxies. Stellar and ionized gas kinematic orientations at every merger stage: We present spatially resolved stellar and/or ionized gas kinematic properties for a sample of 103 interacting galaxies, tracing all merger stages: close companions, pairs with morphological signatures of interaction, and coalesced merger remnants. In order to distinguish kinematic properties caused by a merger event from those driven by internal processes, we compare our galaxies with a control sample of 80 non-interacting galaxies. We measure for both the stellar and the ionized gas components the major (projected) kinematic position angles (PAkin, approaching and receding) directly from the velocity distributions with no assumptions on the internal motions. This method also allow us to derive the deviations of the kinematic PAs from a straight line (delta-PAkin). We find that around half of the interacting objects show morpho-kinematic PA misalignments that cannot be found in the control sample. In particular, we observe those misalignments in galaxies with morphological signatures of interaction. On the other hand, the level of alignment between the approaching and receding sides for both samples is similar, with most of the galaxies displaying small misalignments. Radial deviations of the kinematic PA orientation from a straight line in the stellar component measured by delta-PAkin are large for both samples. However, for a large fraction of interacting galaxies the ionized gas delta-PAkin is larger than the typical values derived from isolated galaxies (48\%), indicating that this parameter is a good indicator to trace the impact of interaction and mergers in the internal motions of galaxies. By comparing the stellar and ionized gas kinematic PA, we find that 42\% (28/66) of the interacting galaxies have misalignments larger than 16 deg., compared to 10\% from the control sample. Our results show the impact of interactions in the motion of stellar and ionized gas as well as the wide the variety of their spatially resolved kinematic distributions.
    \item Deans' book The Radon Transform and some of its applications: \citet{deans2007radon}. Lots of applications and references in here.
\end{itemize}
\subsection{week 23}
\begin{itemize}
    \item Kolgmogorov-Smirnov test: \citet{hodges1958significance}.
\end{itemize}

\section{May 2019}
\subsection{week 21}
\begin{itemize}
    \item Johann Radon's \citep{radon1917determination} On the determination of functions from their integral values along certain manifolds.
    \item Peter Toft's Radon transform thesis \citep{7910dc8d5b654c90ac4bc94c67d06f01}.
    \item Obtained the correct reference to the MaNGA survey overview \citep{2015ApJ...798....7B}.
\end{itemize}


\subsection{Week 18}
\begin{itemize}
    \item \citet{2017MNRAS.472.1401A} : {Massive post-starburst galaxies at z $>$ 1 are compact proto-spheroids} : We investigate the relationship between the quenching of star formation and the structural transformation of massive galaxies, using a large sample of photometrically selected post-starburst galaxies in the UKIDSS Ultra-Deep Survey field. We find that post-starburst galaxies at high redshift (z $>$ 1) show high Sérsic indices, significantly higher than those of active star-forming galaxies, but with a distribution that is indistinguishable from the old quiescent population. We conclude that the morphological transformation occurs before (or during) the quenching of star formation.  Our findings are consistent with a scenario in which massive passive galaxies are formed from three distinct phases: (1) gas-rich dissipative collapse to very high densities, forming the proto-spheroid, (2) rapid quenching of star formation to create the 'red nugget' with post-starburst features and (3) a gradual growth in size as the population ages, perhaps as a result of minor mergers.
\end{itemize}

%%%%%%%%%%%%%%%%%%%%%%%%%%%%%%%%%%%%%%%%%%%%%%%%%%%
\section{April 2019}
\subsection{Week 17}
\par \citet{2019ApJ...872...76N} : {Accurate Identification of Galaxy Mergers with Imaging} : Since mass ratio has the largest effect on the classification, we create separate classification approaches for minor and major mergers that can be applied to SDSS imaging or adapted for other imaging surveys.
\par \citet{2019arXiv190100856W} : {The Data Analysis Pipeline for the SDSS-IV MaNGA IFU Galaxy Survey: Overview} : This paper describes the first public release of the DAP software and assesses its output provided by the recent SDSS Data Release 15 (DR15). The DAP has focused on measurements that are as close as possible to the data and require minimal model-based interpretations or assumptions for reproduction. For DR15, these measurements include stellar kinematics (velocity and velocity dispersion), emission-line properties (kinematics, fluxes, and equivalent widths), and spectral indices (e.g., D4000 and the Lick indices). We provide an overview of how these measurements are made in the MaNGA DAP, including discussion of its software design, workflow, performance, and output data model. \textbf{Section 7 goes into particular depth with regard to the assessments of the
stellar kinematics}. The python code used to produce this plot and many others
in this paper can be found at \url{https://github.com/sdss/mangadap/tree/master/docs/papers/Overview/scripts}.
\par \citet{2019MNRAS.483..172D} : {SDSS-IV MaNGA: signatures of halo assembly in kinematically misaligned galaxies} : We investigate the relationship of kinematically misaligned galaxies with their large-scale environment, in the context of halo assembly bias. According to numerical simulations, halo age at fixed halo mass is intrinsically linked to the large-scale tidal environment created by the cosmic web. We investigate the relationship between distances to various cosmic web features and present-time gas accretion rate. We select a sub-sample of $\sim$900 central galaxies from the MaNGA survey with \textbf{defined global position angles (PA; angle at which velocity change is greatest) for their stellar and H$\alpha$ gas components} up to a minimum of 1.5 effective radii (Re). : The emphasis is on gas-disc PA misalignment. To identify accreting galaxies, we estimate the two-dimensional (2D) global PA of the stellar and H$\alpha$ gas velocity fields using the FIT\_KINEMATIC\_PA routine outlined in Krajnović et al. (2006). By default this finds the angle corresponding to the bisecting line that has greatest velocity change along it (i.e. the angle of peak rotational velocity). 
The offset angle between kinematic components is defined as  $\Delta{PA}=|PA_{stellar}−PA_{H\alpha}|$.
We define galaxies with $\Delta$PA $\ge 30$ degrees to be significantly kinematically misaligned. An example of an aligned and a misaligned galaxy is shown in Fig. 2.
\par \citet{1996ApJ...466..104Z} : {The Environment of ``E+A'' Galaxies} : The spectrum of an "E + A" galaxy (Dressier \& Gunn) which is dominated by a young stellar component but lacks the emission lines characteristic of any significant, on-going star formation suggests that the galaxy experienced a brief, powerful starburst within the last gigayear (Dressler \& Gunn; Couch \& Sharples). We conclude that interactions with the cluster environment, in the form of the intracluster medium or cluster potential, are not essential for "E+A" formation and therefore that the presence of these galaxies in distant clusters does not provide strong evidence for the effects of cluster environment on galaxy evolution. If one mechanism is responsible for "E+A" formation, then the observations that "E+A"s exist in the field and that at least five of the 21 in our sample have clear tidal features argue that galaxy-galaxy interactions and mergers are that mechanism. 
\par \citet{2017MNRAS.466..798C} : {Improving the full spectrum fitting method: accurate convolution with Gauss-Hermite functions} : Penalized Pixel-Fitting (pPXF) - \textbf{Extract the stellar or gas kinematics and stellar population from galaxy spectra via full spectrum fitting}.

\par \citet{2004PASP..116..138C} : {Parametric Recovery of Line-of-Sight Velocity Distributions from Absorption-Line Spectra of Galaxies via Penalized Likelihood} : The dynamics of stars in a galaxy is uniquely defined by the distribution function and the potential in which the stars move. Considering galaxies as pure stellar systems, the spectrum observed at a certain sky position is a (luminosity‐weighted) sum of individual stellar spectra redshifted according to their LOS velocities. If one makes the assumption that the spectrum of all stars is given by a single template, then it simply reduces to the convolution between that spectrum and the LOSVD, which can then be retrieved by solving the inverse problem, i.e., deconvolving the spectra using the template. 
\par \citet{2005MNRAS.357..937G} : {266 E+A galaxies selected from the Sloan Digital Sky Survey Data Release 2: the origin of E+A galaxies}. E+A galaxies are characterized as galaxies with strong Balmer absorption lines but without any [OII] or H$\alpha$ emission lines. The existence of strong Balmer absorption lines indicates that E+A galaxies have experienced starburst within the past one gigayear. However, the lack of [OII] and H$\alpha$ emission lines indicates that E+A galaxies do not have any on-going star formation. Therefore, E+A galaxies are interpreted as post-starburst galaxies. For many years, however, it has been a mystery why E+A galaxies started starburst and why they quenched star formation abruptly. 
\par \citet{10.1093/pasj/55.4.771} : {H$\delta$-Strong Galaxies in the Sloan Digital Sky Survey: I. The Catalog} : We present here a new and homogeneous sample of 3340 galaxies selected from the Sloan Digital Sky Survey (SDSS) \textbf{based solely on the observed strength of their H$\delta$ hydrogen Balmer absorption line}. The presence of a strong H$\delta$ line within the spectrum of a galaxy indicates that the galaxy has undergone a significant change in its star-formation history within the last Gigayear. : \textbf{H$\delta$-strong (HDS) galaxies}.
\subsection{Week 16}
\par \citet{10.1093/mnras/stv2878} : {\textbf{Shape asymmetry:} a morphological indicator for automatic detection of galaxies in the post-coalescence merger stages} : We present a \textbf{new morphological indicator} designed for automated recognition of galaxies with faint asymmetric tidal features suggestive of an ongoing or past merger. We use the new indicator, together with pre-existing diagnostics of galaxy structure to study the role of galaxy mergers in inducing (post-) starburst spectral signatures in local galaxies, and investigate whether (post-) starburst galaxies play a role in the build-up of the ‘red sequence’.
\par \citet{2015A&A...582A..21B} : {Tracing kinematic (mis)alignments in CALIFA merging galaxies. Stellar and ionized gas kinematic orientations at every merger stage}. The aim of this work is to analyze the stellar and ionized gas velocity distributions as the merger event evolves by studying several galaxies at different stages of this event; we also want to compare kinematic properties of these galaxies with those derived from a set of non-interacting objects. 
\par \citet{2014MNRAS.438.1038R}: {Caught in the act: cluster `k+a' galaxies as a link between spirals and S0s}: Gentler mechanisms, such as ram-pressure stripping or weak galaxy-galaxy interactions, appear to be responsible for ending star formation in these intermediate-redshift cluster disc galaxies.

\par \citet{2011MNRAS.414.2923K} : {The ATLAS$^{3D}$ project - II. Morphologies, kinemetric features and alignment between photometric and kinematic axes of early-type galaxies}
\par \citet{2007MNRAS.382..960K} : {The UV properties of E+A galaxies: constraints on feedback-driven quenching of star formation}
\par \citet{2018MNRAS.480.2217S} : {SDSS-IV MaNGA: characterizing non-axisymmetric motions in galaxy velocity fields using the Radon transform} :  We demonstrate the \textbf{Radon transform} by applying it to gas and stellar velocity fields from the first $\sim$2800 galaxies of the SDSS-IV MaNGA IFU survey. We separately classify gas and stellar velocity fields into five categories based on the shape of their radial PA$_k$ profiles. 
\par \citet{2003MNRAS.344.1000B} : {Stellar population synthesis at the resolution of 2003} : We present a new model for computing the spectral evolution of stellar populations... The model reproduces in detail typical galaxy spectra from the Early Data Release (EDR) of the Sloan Digital Sky Survey (SDSS). This model should be particularly useful for interpreting the spectra gathered by modern spectroscopic surveys in terms of constraints on the star formation histories and metallicities of galaxies.
\par \citet{2000ApJ...529..886C} : {The Asymmetry of Galaxies: Physical Morphology for Nearby and High-Redshift Galaxies} : Describes how measured asymmetry can be an indicator of galaxy interaction or mergers.

\subsection{Week 15}
\par \citet{2012MNRAS.420..672S} : {From star-forming spirals to passive spheroids: integral field spectroscopy of E+A galaxies} : Abstract: We present three-dimensional spectroscopy of 11 E+A galaxies at z= 0.06-0.12. These galaxies were selected for their strong H$\delta$ [JP: 4102\AA] absorption but weak (or non-existent) [O II] $\lambda$3727 and H$\alpha$ emission. This selection suggests that a recent burst of star formation was triggered but subsequently abruptly ended. The analysis uses H-delta equivalent widths. A stars cover 33 percent of the galaxy image. 

\par \citet{1997A&A...325.1025P} : {Indicators of star formation: 4000 {\r{A}} break and Balmer lines.} : Dn4000 and equivalent width of Balmer lines. Describes the application of D$_n$4000 and H$\delta$ spectral indices as an indication of recent star formation. 
\par \citet{2004ApJ...609..683T} : {Field E+A Galaxies at Intermediate Redshifts ($0.3 < z < 1$)} : PSB galaxies.
\par \citet{2009MNRAS.395..144W} : {Post-starburst galaxies: more than just an interesting curiosity} : {Post-starburst galaxies in the VVDS : PCA of the $4000\AA$ break.} : Are gas-rich mergers an important mechanism for the build-up of the red sequence since $z\sim1$.
\par \citet{2004ApJ...608..752B} : {Nearly 5000 Distant Early-Type Galaxies in COMBO-17: A Red Sequence and Its Evolution since $z\sim1$} : Galaxy colour bi-modality
\par \citet{1972ApJ...176....1G} : {On the Infall of Matter Into Clusters of Galaxies and Some Effects on Their Evolution} : Ram pressure stripping.
\par \citet{2006MNRAS.366..787K} : {Kinemetry: a generalization of photometry to the higher moments of the line-of-sight velocity distribution} : The wealth of features seen in stellar kinematic maps of early-type galaxies (Emsellem et al. 2004) confirms the usefulness of two-dimensional data, but also poses a problem to efficiently harvest and interpret the important features from the maps.
\subsection{Week 14}
\par \citet{2011ApJ...742...11S} The Interstellar Medium in Distant Star-forming Galaxies: Turbulent Pressure, Fragmentation, and Cloud Scaling Relations in a Dense Gas Disk at z = 2.3.
\par \citet{brian_cherinka_2018_1146705} : sdss/marvin: Marvin Beta 2.2.0
\par \citet{2008ApJ...682..231S} : {Kinemetry of SINS High-Redshift Star-Forming Galaxies: Distinguishing Rotating Disks from Major Mergers}.
\par \citet{2016A&A...591A..85B} : {Distinguishing disks from mergers: Tracing the kinematic asymmetries in local (U)LIRGs using kinemetry-based criteria}.

\section{March 2019}
\citet{2016MNRAS.463..832W} : The evolution of post-starburst galaxies from z=2 to 0.5
\par \citet{2017MNRAS.472.1401A} : Massive post-starburst galaxies at $z \ge 1$ are compact proto-spheroids

\section{February 2019}
\citet{Cappellari2008} : {Measuring the inclination and mass-to-light ratio of
        axisymmetric galaxies via anisotropic Jeans models of stellar
        kinematics}
\par \citet{Bundy_2014} : {{OVERVIEW} {OF} {THE} {SDSS}-{IV} {MaNGA} {SURVEY}: {MAPPING} {NEARBY} {GALAXIES} {AT} {APACHE} {POINT} {OBSERVATORY}}


%%%%%%%%%%%%%%%%%%%% REFERENCES %%%%%%%%%%%%%%%%%%
% The best way to enter references is to use BibTeX:
\bibliographystyle{mnras}
\bibliography{JPbib2019} 

\end{document}
